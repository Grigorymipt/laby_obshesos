\documentclass[a4paper,12pt]{article}

\usepackage[T2A]{fontenc}
\usepackage[utf8]{inputenc}
\usepackage[english,russian]{babel}

\usepackage[left = 2cm, right = 2cm, bottom= 2cm, top =2cm]{geometry}

\usepackage{amsmath,amsfonts,amssymb,amsthm,mathtools}
\usepackage{graphicx}
\usepackage{wrapfig}

\author{Ковешников Григорий}
\title{Определение модуля Юнга на основе исследевания деформациия деформаций растяжения и изгиба}
\date{\today}
\begin{document}
	\maketitle
	\textbf{Цель работы:} экспериспериментально получить зависимость между напряжением и деформацией(Закон Гука) для двух тел: одноосного растяжения и чистого изгиба; по резудьтатам измерений вычислить модуль Юнга.\\
	
	Работу можно разделить на две части: в первой части наблюдаем растяжение проволоки под действием грузов разной массы, во второй части рассматриваем стерженинь, изгибаемый под действием силы, действующей перпендикулярно одной из его сторон, измеряя зависимость удлинения от массы находим модуль Юнга для обеих частей эксперимента. \\ \\
	\large{\textbf{Определение модуля Юнга по измерениям растяжения проволоки:}}\\
	В ходе эксперисмента используем прибор Лермантова. % Полить воды про прибор
	Измерив все параметры установки занесем их значения в таблицу \ref{tabl:1}.
	
	\begin{table}
		\caption{Параметры установки:}
		\label{tabl:1}
		\begin{center}
			\begin{tabular}{|l|l|l|l|l|}\hline
				\(l_{\parallel}\), мм & \(l_{\perp}\), мм & \(r\), мм & \(d_{wire}\), мм \\ \hline
				134.0 \(\pm 1\) & 176.3 \(\pm 1\) & 15 & 0.46 \\ \hline	
				
			\end{tabular}
		\end{center}
	\end{table}
	
	После настройки оптического прибора и наведения его на шкалу выведем формулу для зависимости значений на шкале от деформации проволоки, учитывая конструкцию установки.%Тут должна быть формула.
	Вычислим максимально допустимую массу грузов, учитывая напряжение разрыва проволои:%Тут вычисления напряжения и какая то формула.
	Учитывая то, что общая массы всех грузов равна:%чему? 
	, и то, максимально допустимая масса равна:%чему?
	То эксперимент выполняется для любого количества грузов из имеющихся. Теперь увеличивая количество грузов по одному будем измерять изменение длины на шкале и дойдя до максимального количества, будем снимать по одному грузу и также измерять показания на шкале, проведем такой эксперимент 3 раза и занесем данные в таблицу,:
	
	\begin{table}
		\caption{Измерения первой части:}
		\begin{center}
			\begin{tabular}{|l|l|l|l|l|l|l|l|l|l|}
				\hline 
				m, г & 0 & 245.2 & 490.5 & 736.1 & 980.5 & 1226.6 & 1472.2 & 1717.7 & 1963.2  \\ 
				\hline
				l, мм & 15.0 & 17.7 & 20.2 & 22.6 & 25.1 & 27.1 & 29.8 & 32.2 & 34.5    \\ 
				\hline
				m, г & 2208.8 & 1963.2 & 1717.7 & 1472.2 & 1226.6 & 980.5 & 736.1 & 490.5 & 245.2 \\
				\hline
				
				l, мм & 36.7 & 34.5 & 32.2 & 29.7 & 27.5 & 25.1 & 22.7 & 20.2 & 17.6 \\
				\hline
				
				m, г & 0 & 245.2 & 490.5 & 736.1 & 980.5 & 1226.6 & 1472.2 & 1717.7 & 1963.2  \\ 
				\hline
				
				l, мм & 15.0 & 17.5 & 20.2 & 22.7 & 24.9 & 27.3 & 29.7 & 32.0 & 34.3 \\
				\hline
				
				m, г & 2208.8 & 1963.2 & 1717.7 & 1472.2 & 1226.6 & 980.5 & 736.1 & 490.5 & 245.2 \\
				\hline
				
				l, мм & 36.7 & 34.4 & 31.9 & 29.5 & 27.1 & 24.7 & 22.2 & 19.8 & 17.2 \\
				\hline
				
				m, г & 0 & 245.2 & 490.5 & 736.1 & 980.5 & 1226.6 & 1472.2 & 1717.7 & 1963.2  \\ 
				\hline
				
				l, мм & 14.5 & 17.2 & 19.7 & 22.2 & 24.6 & 27.0 & 29.3 & 31.7 & 33.3 \\
				\hline
				
				m, г & 2208.8 & 1963.2 & 1717.7 & 1472.2 & 1226.6 & 980.5 & 736.1 & 490.5 & 245.2 \\
				\hline
				
				l, мм & 36.2 & 34.0 & 31.7 & 29.4 & 27.0 & 24.6 & 22.1 & 19.7 & 17.1 \\
				\hline
				
			\end{tabular}
		\end{center}
	\end{table}
	В силу того, что изначально на проволоке находилось 2 груза, то проволока была напряжена, значит на первых грузах удлиннение будет расти нормально. Построим график функции:
	%\includegraphics.......
	
		

\large{\textbf{Определение модуля Юнга по измерениям изгиба балки:}}\\
Измеряем параметры установки и занесем данные в таблицу:
\begin{table}
\caption{Параметры установки:}
\begin{center}
\begin{tabular}{|c|c|c|c|c|c|c|}

\hline 
\rule[-1ex]{0pt}{2.5ex} \(l_{AB}\), мм & \(a_1\), мм & \(b_1\), мм & \(a_2\), мм & \(b_2\), мм & \(a_3\), мм & \(b_3\), мм \\ 
\hline 
\rule[-1ex]{0pt}{2.5ex} \(500 \pm 1\) & \( \pm 1\) & \( \pm 1\) & \( \pm 1\) & \( \pm 1\) & \( \pm 1\) & \( \pm 1\) \\ 
\hline 
\end{tabular}
\end{center}
\end{table} 
Под дейтвием грузов разной массы измеряем изменение центральной точки балки под действие перпендикулярной силы, измерения будем проводить с помощью индикатора, расположим брусок точно посередине между призм установки. Из теоретических соображений был выведен закон, по которому находим модуль Юнга:
\[E = \dfrac{Pl^3}{4 a b^3 y_{max}}\]
Проведем эксперимент зависимости силы от смещения точки ее приложения для балки 2 для одного груза:\\
Занесем данные в таблицу:\\

\begin{tabular}{|c|c|c|c|c|c|c|c|c|c|}
\hline 
\( \Delta l_x \), мм & 0 & 2.5 & 5.0 & 7.5 & 10.0 & 12.5 & 15.0 & 17.5 & 20.0 \\ 
\hline 
\( \Delta l_y \), мм & 11.19 & 11.26 & 11.16 & 11.20 & 11.19 & 11.18 & 11.19 & 11.21 & 11.21 \\ 
\hline 
\end{tabular}

Понятно, что изменения силы в малых (около сантиметра) пределах не существенно, по сравнению со случайной погрешностью измерения прибора. 
Занесем данные измерений зависимоти массы грузов в таблицу, для трех балок:\\

 

Постоим 3 графика \(P(y_{max})\) для каждого груза:\\

%graphics here
Тогда найдем модуль Юнга для каждого бруска по наклону графика и погрешности:\\
\(P = \left(E\cdot \dfrac{4ab^3}{l^3}\right)\cdot y_{max}\);
\(k = \left(E\cdot \dfrac{4ab^3}{l^3}\right)\)\\ 
C помощью метода наименьших квадратов найдем найдем k:\\
\(k = \dfrac{\langle P y_{max}\rangle}{\langle y_{max}^2\rangle}\)\\ Случайная погрешность: 
\(\sigma_{rand} = \dfrac{1}{\sqrt{n}}\sqrt{\dfrac{\langle P^2 \rangle}{\langle y_{max}^2 \rangle} - k^2}\); \(\sigma_{rand} =\) \\
Систематичиская погрешность: \\
\(\varepsilon_{syst} = \sqrt{\left( \dfrac{\sigma_{a}}{a} \right)^2 + \left(3 \dfrac{\sigma_{b}}{b} \right)^2 + \left( 3\dfrac{\sigma_{l}}{l} \right)^2 + \left( \dfrac{\sigma_{y_{max}}}{y_{max}} \right)^2 + \left( \dfrac{\sigma_{P}}{P} \right)^2}\)\\
\( \varepsilon_{syst} = \)\\
Т.к. \( E = \dfrac{kl^3}{ab^3y_{max}} \), то 
\( \sigma_E = \sqrt{ \sigma_{rand}^2 + ( \varepsilon_{syst}E )^2} \);
\(\sigma_E = \)\\
\(E = ( \pm )\)
Табличные знаяения модуля Юнга для латуни и дерева равны соответственно: \(E_w = \), \(E_l = \). Сранвнивая полученные результаты с табличными знаяениями получаем, что в пределах погрешности результаты равны табличным. \\
\textbf{Вывод:} из обеих частей эксперимента применяем способы нахождения модуля юнга для упругих тел и получаем равенство в пределах погрешностей.



\end{document}
